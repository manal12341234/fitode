%\VignetteEngine{knitr::knitr}
%\VignetteDepends{fitode}
%\VignetteIndexEntry{Getting started with \texttt{fitode} package}
\documentclass{article}\usepackage[]{graphicx}\usepackage[]{color}
%% maxwidth is the original width if it is less than linewidth
%% otherwise use linewidth (to make sure the graphics do not exceed the margin)
\makeatletter
\def\maxwidth{ %
  \ifdim\Gin@nat@width>\linewidth
    \linewidth
  \else
    \Gin@nat@width
  \fi
}
\makeatother

\definecolor{fgcolor}{rgb}{0.345, 0.345, 0.345}
\newcommand{\hlnum}[1]{\textcolor[rgb]{0.686,0.059,0.569}{#1}}%
\newcommand{\hlstr}[1]{\textcolor[rgb]{0.192,0.494,0.8}{#1}}%
\newcommand{\hlcom}[1]{\textcolor[rgb]{0.678,0.584,0.686}{\textit{#1}}}%
\newcommand{\hlopt}[1]{\textcolor[rgb]{0,0,0}{#1}}%
\newcommand{\hlstd}[1]{\textcolor[rgb]{0.345,0.345,0.345}{#1}}%
\newcommand{\hlkwa}[1]{\textcolor[rgb]{0.161,0.373,0.58}{\textbf{#1}}}%
\newcommand{\hlkwb}[1]{\textcolor[rgb]{0.69,0.353,0.396}{#1}}%
\newcommand{\hlkwc}[1]{\textcolor[rgb]{0.333,0.667,0.333}{#1}}%
\newcommand{\hlkwd}[1]{\textcolor[rgb]{0.737,0.353,0.396}{\textbf{#1}}}%
\let\hlipl\hlkwb

\usepackage{framed}
\makeatletter
\newenvironment{kframe}{%
 \def\at@end@of@kframe{}%
 \ifinner\ifhmode%
  \def\at@end@of@kframe{\end{minipage}}%
  \begin{minipage}{\columnwidth}%
 \fi\fi%
 \def\FrameCommand##1{\hskip\@totalleftmargin \hskip-\fboxsep
 \colorbox{shadecolor}{##1}\hskip-\fboxsep
     % There is no \\@totalrightmargin, so:
     \hskip-\linewidth \hskip-\@totalleftmargin \hskip\columnwidth}%
 \MakeFramed {\advance\hsize-\width
   \@totalleftmargin\z@ \linewidth\hsize
   \@setminipage}}%
 {\par\unskip\endMakeFramed%
 \at@end@of@kframe}
\makeatother

\definecolor{shadecolor}{rgb}{.97, .97, .97}
\definecolor{messagecolor}{rgb}{0, 0, 0}
\definecolor{warningcolor}{rgb}{1, 0, 1}
\definecolor{errorcolor}{rgb}{1, 0, 0}
\newenvironment{knitrout}{}{} % an empty environment to be redefined in TeX

\usepackage{alltt}
\title{Getting started with the \texttt{fitode} package}
\author{Sang Woo Park}
\usepackage{amsmath}
\usepackage{natbib}
\usepackage{hyperref}
\newcommand{\rzero}{{\cal R}_0}
\newcommand{\code}[1]{{\tt #1}}
\newcommand{\bmb}[1]{{\color{blue} bmb: \emph{#1}}}
\bibliographystyle{chicago}
\date{\today}
\IfFileExists{upquote.sty}{\usepackage{upquote}}{}
\begin{document}
\maketitle



\tableofcontents

\pagebreak

\section{Introduction}

\texttt{fitode} is an R package for fitting ordinary differential equations (ODE)
using Maximum Likelihood or Bayesian Markov Chain Monte Carlo (MCMC). It relies on
automatic differentiation features of the \texttt{Deriv} package to solve the
sensitivity equations and use gradient-based optimization algorithms.
\begin{itemize}
    \item response distributions: Gamma, Gaussian, Poisson, and negative
            binomial (NB1 and NB2 parameterization)
    \item link functions on model parameters: log, logit, and identity
    \item fitting multiple state to multivariate time series
    \item prior/penalization: Beta, Gamma, and Gaussian distributions
    \item confidence intervals on parameters and their transformations via
            delta method, profiling, and importance sampling
\end{itemize}

In order to construct a model in \texttt{fitode} you need to:
\begin{itemize}
    \item specify the gradients using formula notation (e.g., $dX/dt=f(X)$ is
            expressed as \texttt{X$\sim$f(X)})
    \item specify the observation process using formula notation (e.g.,
            \texttt{Xobs$\sim$dnorm(mean=X, sd=sigma)})
    \item specify the initial conditions using formula notation
    \item specify the parameters of the model
    \item specify the link functions (log-link is the default)
\end{itemize}
To fit a model, you need to:
\begin{itemize}
    \item specify the data (as well as the time column)
    \item specify the starting values for optimization or MCMC
    \item optionally specify fixed parameters
    \item optionally specify prior distributions (or penalizations); not
            specifying prior distribution in MCMC will result in improper priors on link scales
\end{itemize}
This document was generated using R version 3.5.1 (2018-07-02) and package versions:
\begin{knitrout}
\definecolor{shadecolor}{rgb}{0.969, 0.969, 0.969}\color{fgcolor}\begin{kframe}
\begin{verbatim}
##   bbmle   Deriv deSolve  fitode ggplot2 
##  1.0.20   3.8.5    1.21   0.1.0   3.1.0
\end{verbatim}
\end{kframe}
\end{knitrout}

\section{Basic fitting - estimating epidemic growth rates}

\subsection{Data}

Here, we study a time series of confirmed cases of Ebola during the 2014 outbreak in
Sierra Leone to characterize epidemic growth patterns. Once you load \texttt{fitode},
the data set (\texttt{SierraLeone2014}) will be automatically loaded to the global
environment.
\begin{knitrout}
\definecolor{shadecolor}{rgb}{0.969, 0.969, 0.969}\color{fgcolor}\begin{kframe}
\begin{alltt}
\hlkwd{library}\hlstd{(ggplot2);} \hlkwd{theme_set}\hlstd{(}\hlkwd{theme_bw}\hlstd{())}
\hlkwd{library}\hlstd{(fitode)}
\hlkwd{plot}\hlstd{(SierraLeone2014)}
\end{alltt}
\end{kframe}
\includegraphics[width=\maxwidth]{figure/SierraLeonedata-1} 

\end{knitrout}

\subsection{Exponential growth model}

Exponential growth model is one of the simplest differential models we can use to
characterize the initial spread of a disease:
\begin{equation}
\frac{dX}{dt} = rX.
\end{equation}
This model is parameterized by the initial growth rate $r$ and the initial value $X(0)$.
Variable $X$ describes the dynamics of \emph{mean} confirmed cases;
for simplicity, we can assume that the observed number of confirmed cases at time $t$ follows a
Poisson error distribution with mean $X(t)$. This model can be constructed in \texttt{fitode} as
follows:
\begin{knitrout}
\definecolor{shadecolor}{rgb}{0.969, 0.969, 0.969}\color{fgcolor}\begin{kframe}
\begin{alltt}
\hlstd{exp_model} \hlkwb{<-} \hlkwd{new}\hlstd{(}\hlstr{"model.ode"}\hlstd{,}
    \hlkwc{name}\hlstd{=}\hlstr{"exponential"}\hlstd{,}
    \hlkwc{model}\hlstd{=}\hlkwd{list}\hlstd{(}
        \hlstd{X} \hlopt{~} \hlstd{r} \hlopt{*} \hlstd{X}
    \hlstd{),}
    \hlkwc{observation}\hlstd{=}\hlkwd{list}\hlstd{(}
        \hlstd{confirmed} \hlopt{~} \hlkwd{dpois}\hlstd{(}\hlkwc{lambda}\hlstd{=X)}
    \hlstd{),}
    \hlkwc{initial}\hlstd{=}\hlkwd{list}\hlstd{(}
        \hlstd{X} \hlopt{~} \hlstd{X0}
    \hlstd{),}
    \hlkwc{par}\hlstd{=}\hlkwd{c}\hlstd{(}\hlstr{"r"}\hlstd{,} \hlstr{"X0"}\hlstd{)}
\hlstd{)}
\end{alltt}
\end{kframe}
\end{knitrout}
Note that the name of the observed variable (\texttt{confirmed}) must be different
from the name of the state variables (e.g., \texttt{X}) because \texttt{fitode}
relies on expression evalulations.

In order to fit this model to the data, we have to specify starting parameter for
the optimization.
To do so, we can simulate the model for various parameters and try to find a reasonable
parameter set by eyes. For example, here is a parameter set that I found by trial and error:
\begin{knitrout}
\definecolor{shadecolor}{rgb}{0.969, 0.969, 0.969}\color{fgcolor}\begin{kframe}
\begin{alltt}
\hlstd{start} \hlkwb{<-} \hlkwd{c}\hlstd{(}\hlkwc{r}\hlstd{=}\hlnum{7}\hlstd{,} \hlkwc{X0}\hlstd{=}\hlnum{30}\hlstd{)}
\hlstd{ss} \hlkwb{<-} \hlkwd{simulate}\hlstd{(exp_model,} \hlkwc{parms}\hlstd{=start,} \hlkwc{times}\hlstd{=SierraLeone2014}\hlopt{$}\hlstd{times)}
\hlkwd{plot}\hlstd{(SierraLeone2014)}
\hlkwd{lines}\hlstd{(X} \hlopt{~} \hlstd{times,} \hlkwc{data}\hlstd{=ss)}
\hlkwd{abline}\hlstd{(}\hlkwc{v}\hlstd{=}\hlnum{2014.8}\hlstd{,} \hlkwc{col}\hlstd{=}\hlstr{"red"}\hlstd{,} \hlkwc{lty}\hlstd{=}\hlnum{2}\hlstd{)}
\end{alltt}
\end{kframe}
\includegraphics[width=\maxwidth]{figure/expsetup-1} 

\end{knitrout}
Here, I used the \texttt{simulate} function to simulate the model. It requires a
parameter set (\texttt{parms} argument) and a time vector (\texttt{times} argument)
to run. Then, it returns a numerical solution for each state variable as well
as simulated observations;
we will ignore the simulated observations for now.

The data does not exhibit exponential growth forever. In order to fit the exponential
model, we have to determine a fitting window. For brevity, we will fit the model from the
beginning of an epidemic to 2014.8 (red dashed line in the previous figure).
\begin{knitrout}
\definecolor{shadecolor}{rgb}{0.969, 0.969, 0.969}\color{fgcolor}\begin{kframe}
\begin{alltt}
\hlstd{exp_fit} \hlkwb{<-} \hlkwd{fitode}\hlstd{(}
    \hlkwc{model}\hlstd{=exp_model,}
    \hlkwc{data}\hlstd{=SierraLeone2014[SierraLeone2014}\hlopt{$}\hlstd{times} \hlopt{<+} \hlnum{2014.8}\hlstd{,],}
    \hlkwc{start}\hlstd{=start}
\hlstd{)}
\end{alltt}


{\ttfamily\noindent\itshape\color{messagecolor}{\#\# Fitting ode ...}}

{\ttfamily\noindent\itshape\color{messagecolor}{\#\# Computing vcov on the original scale ...}}\end{kframe}
\end{knitrout}
We can see that the estimated parameters are very close to our initial guess:
\begin{knitrout}
\definecolor{shadecolor}{rgb}{0.969, 0.969, 0.969}\color{fgcolor}\begin{kframe}
\begin{alltt}
\hlstd{exp_fit}
\end{alltt}
\begin{verbatim}
## Model: exponential 
## 
## Observations:
## confirmed ~ dpois(lambda = X) 
## 
## Coefficients:
##         r        X0 
##  6.993036 30.249604 
## 
## Log-Likelihood:-140.07 
## 
## link: r = log; X0 = log
\end{verbatim}
\end{kframe}
\end{knitrout}
We can quantify the uncertainty in the parameters by using \texttt{confint}:
\begin{knitrout}
\definecolor{shadecolor}{rgb}{0.969, 0.969, 0.969}\color{fgcolor}\begin{kframe}
\begin{alltt}
\hlkwd{confint}\hlstd{(exp_fit)}
\end{alltt}
\begin{verbatim}
##     estimate     2.5 %    97.5 %
## r   6.993036  6.652687  7.350797
## X0 30.249604 27.308713 33.507203
\end{verbatim}
\end{kframe}
\end{knitrout}
By default, \texttt{confint} will calculate the confidence intervals using the delta method.
We diagnose the fit by using the \texttt{plot} function:
\begin{knitrout}
\definecolor{shadecolor}{rgb}{0.969, 0.969, 0.969}\color{fgcolor}\begin{kframe}
\begin{alltt}
\hlkwd{plot}\hlstd{(exp_fit,} \hlkwc{level}\hlstd{=}\hlnum{0.95}\hlstd{)}
\end{alltt}
\end{kframe}
\includegraphics[width=\maxwidth]{figure/unnamed-chunk-1-1} 

\end{knitrout}

We can see that the uncertainty of our fit is too narrow. This is likely to be
driven by our choice of the error function. Poisson distribution assumes that variance
is equal to the mean. Instead, we can use a negative binomial distribution, which
assumes that variance is a quadratic function of the mean. Then, we have to estimate an
extra parameter (\texttt{size} argument of the \texttt{dnbinom}) to account for overdispersion:
\begin{knitrout}
\definecolor{shadecolor}{rgb}{0.969, 0.969, 0.969}\color{fgcolor}\begin{kframe}
\begin{alltt}
\hlstd{exp_model_nbinom} \hlkwb{<-} \hlkwd{new}\hlstd{(}\hlstr{"model.ode"}\hlstd{,}
    \hlkwc{name}\hlstd{=}\hlstr{"exponential (nbinom)"}\hlstd{,}
    \hlkwc{model}\hlstd{=}\hlkwd{list}\hlstd{(}
        \hlstd{X} \hlopt{~} \hlstd{r} \hlopt{*} \hlstd{X}
    \hlstd{),}
    \hlkwc{observation}\hlstd{=}\hlkwd{list}\hlstd{(}
        \hlstd{confirmed} \hlopt{~} \hlkwd{dnbinom}\hlstd{(}\hlkwc{mu}\hlstd{=X,} \hlkwc{size}\hlstd{=phi)}
    \hlstd{),}
    \hlkwc{initial}\hlstd{=}\hlkwd{list}\hlstd{(}
        \hlstd{X} \hlopt{~} \hlstd{X0}
    \hlstd{),}
    \hlkwc{par}\hlstd{=}\hlkwd{c}\hlstd{(}\hlstr{"r"}\hlstd{,} \hlstr{"X0"}\hlstd{,} \hlstr{"phi"}\hlstd{)}
\hlstd{)}
\end{alltt}
\end{kframe}
\end{knitrout}
We can fit the model again:
\begin{knitrout}
\definecolor{shadecolor}{rgb}{0.969, 0.969, 0.969}\color{fgcolor}\begin{kframe}
\begin{alltt}
\hlstd{exp_fit_nbinom} \hlkwb{<-} \hlkwd{fitode}\hlstd{(}
    \hlkwc{model}\hlstd{=exp_model_nbinom,}
    \hlkwc{data}\hlstd{=SierraLeone2014[SierraLeone2014}\hlopt{$}\hlstd{times} \hlopt{<+} \hlnum{2014.8}\hlstd{,],}
    \hlkwc{start}\hlstd{=}\hlkwd{c}\hlstd{(start,} \hlkwc{phi}\hlstd{=}\hlnum{10}\hlstd{)}
\hlstd{)}
\end{alltt}


{\ttfamily\noindent\itshape\color{messagecolor}{\#\# Fitting ode ...}}

{\ttfamily\noindent\itshape\color{messagecolor}{\#\# Computing vcov on the original scale ...}}\end{kframe}
\end{knitrout}
Note that we need to specify a starting value for the overdispersion parameter as
well.

We can plot this fit:
\begin{knitrout}
\definecolor{shadecolor}{rgb}{0.969, 0.969, 0.969}\color{fgcolor}\begin{kframe}
\begin{alltt}
\hlkwd{plot}\hlstd{(exp_fit_nbinom,} \hlkwc{level}\hlstd{=}\hlnum{0.95}\hlstd{)}
\end{alltt}
\end{kframe}
\includegraphics[width=\maxwidth]{figure/unnamed-chunk-4-1} 

\end{knitrout}
We can see that our uncertainty is more reasonable. This
increases confidence intervals on parameters as well:
\begin{knitrout}
\definecolor{shadecolor}{rgb}{0.969, 0.969, 0.969}\color{fgcolor}\begin{kframe}
\begin{alltt}
\hlkwd{confint}\hlstd{(exp_fit_nbinom)}
\end{alltt}
\begin{verbatim}
##      estimate     2.5 %    97.5 %
## r    7.589514  6.500229  8.861336
## X0  26.326482 20.089257 34.500213
## phi 12.903678  5.266959 31.613098
\end{verbatim}
\end{kframe}
\end{knitrout}

\subsection{Logistic growth model}

Exponential growth model accounts for only the initial portion of
the observed data. Instead, we might want to try to model the entire
time series. Note that the cumulative number of cases saturate over time:
\begin{knitrout}
\definecolor{shadecolor}{rgb}{0.969, 0.969, 0.969}\color{fgcolor}\begin{kframe}
\begin{alltt}
\hlkwd{plot}\hlstd{(}\hlkwd{cumsum}\hlstd{(confirmed)} \hlopt{~} \hlstd{times,} \hlkwc{data}\hlstd{=SierraLeone2014)}
\end{alltt}
\end{kframe}
\includegraphics[width=\maxwidth]{figure/unnamed-chunk-6-1} 

\end{knitrout}
We can use a logistic model to describe this saturating patrern:
\begin{equation}
\frac{dX}{dt} = r X \left(1 - \frac{X}{K}\right).
\end{equation}
While we can fit $X$ directly to cumulative number of cases, it can lead to
overly confident results due to accumulation of observation error \citep{king2015avoidable}.
Instead, we can use \emph{interval counts} to model the true number of cases:
$X(t) - X(t - \Delta t)$, where $\Delta t$ is the reporting time step.
This is done by using the \texttt{diffnames} argument
\begin{knitrout}
\definecolor{shadecolor}{rgb}{0.969, 0.969, 0.969}\color{fgcolor}\begin{kframe}
\begin{alltt}
\hlstd{logistic_model} \hlkwb{<-} \hlkwd{new}\hlstd{(}\hlstr{"model.ode"}\hlstd{,}
    \hlkwc{name}\hlstd{=}\hlstr{"logistic (nbinom)"}\hlstd{,}
    \hlkwc{model}\hlstd{=}\hlkwd{list}\hlstd{(}
        \hlstd{X} \hlopt{~} \hlstd{r} \hlopt{*} \hlstd{X} \hlopt{*} \hlstd{(}\hlnum{1} \hlopt{-} \hlstd{X}\hlopt{/}\hlstd{K)}
    \hlstd{),}
    \hlkwc{observation}\hlstd{=}\hlkwd{list}\hlstd{(}
        \hlstd{confirmed} \hlopt{~} \hlkwd{dnbinom}\hlstd{(}\hlkwc{mu}\hlstd{=X,} \hlkwc{size}\hlstd{=phi)}
    \hlstd{),}
    \hlkwc{initial}\hlstd{=}\hlkwd{list}\hlstd{(}
        \hlstd{X} \hlopt{~} \hlstd{X0}
    \hlstd{),}
    \hlkwc{diffnames}\hlstd{=}\hlstr{"X"}\hlstd{,}
    \hlkwc{par}\hlstd{=}\hlkwd{c}\hlstd{(}\hlstr{"r"}\hlstd{,} \hlstr{"X0"}\hlstd{,} \hlstr{"K"}\hlstd{,} \hlstr{"phi"}\hlstd{)}
\hlstd{)}
\end{alltt}
\end{kframe}
\end{knitrout}

In this case, we need to modify the data set by adding an extra \texttt{NA}
observation before the first observation; this allows \texttt{fitode} to take
the interval difference and still end up with the same number of observations
as the time series.
\begin{knitrout}
\definecolor{shadecolor}{rgb}{0.969, 0.969, 0.969}\color{fgcolor}\begin{kframe}
\begin{alltt}
\hlstd{SierraLeone2014b} \hlkwb{<-} \hlkwd{rbind}\hlstd{(}
    \hlkwd{c}\hlstd{(}\hlkwc{times}\hlstd{=SierraLeone2014}\hlopt{$}\hlstd{times[}\hlnum{1}\hlstd{]} \hlopt{-}
          \hlkwd{diff}\hlstd{(SierraLeone2014}\hlopt{$}\hlstd{times)[}\hlnum{1}\hlstd{],} \hlkwc{confirmed}\hlstd{=}\hlnum{NA}\hlstd{),}
    \hlstd{SierraLeone2014}
\hlstd{)}
\end{alltt}
\end{kframe}
\end{knitrout}

Again, we can try to find a reasonable parameter set by trial and error:
\begin{knitrout}
\definecolor{shadecolor}{rgb}{0.969, 0.969, 0.969}\color{fgcolor}\begin{kframe}
\begin{alltt}
\hlstd{start_logistic} \hlkwb{<-}
    \hlkwd{c}\hlstd{(}\hlkwd{coef}\hlstd{(exp_fit_nbinom),} \hlkwc{K}\hlstd{=}\hlkwd{sum}\hlstd{(SierraLeone2014}\hlopt{$}\hlstd{confirmed))}
\hlcom{## need to use a different value for X0}
\hlstd{start_logistic[[}\hlstr{"X0"}\hlstd{]]} \hlkwb{<-} \hlnum{300}
\hlstd{ss_logistic} \hlkwb{<-} \hlkwd{simulate}\hlstd{(}
    \hlstd{logistic_model,}
    \hlkwc{parms}\hlstd{=start_logistic,}
    \hlkwc{times}\hlstd{=SierraLeone2014b}\hlopt{$}\hlstd{times}
\hlstd{)}

\hlkwd{plot}\hlstd{(SierraLeone2014)}
\hlkwd{lines}\hlstd{(X}\hlopt{~}\hlstd{times,} \hlkwc{data}\hlstd{=ss_logistic)}
\end{alltt}
\end{kframe}
\includegraphics[width=\maxwidth]{figure/unnamed-chunk-9-1} 

\end{knitrout}
and fit the model:
\begin{knitrout}
\definecolor{shadecolor}{rgb}{0.969, 0.969, 0.969}\color{fgcolor}\begin{kframe}
\begin{alltt}
\hlstd{logistic_fit} \hlkwb{<-} \hlkwd{fitode}\hlstd{(}
    \hlstd{logistic_model,}
    \hlkwc{data}\hlstd{=SierraLeone2014b,}
    \hlkwc{start}\hlstd{=start_logistic}
\hlstd{)}
\end{alltt}


{\ttfamily\noindent\itshape\color{messagecolor}{\#\# Fitting ode ...}}

{\ttfamily\noindent\itshape\color{messagecolor}{\#\# Computing vcov on the original scale ...}}\end{kframe}
\end{knitrout}
In this case, we get a much higer growth rate estimate:
\begin{knitrout}
\definecolor{shadecolor}{rgb}{0.969, 0.969, 0.969}\color{fgcolor}\begin{kframe}
\begin{alltt}
\hlkwd{confint}\hlstd{(logistic_fit)}
\end{alltt}
\begin{verbatim}
##        estimate       2.5 %       97.5 %
## r      9.404301    8.879291     9.960355
## X0   123.985064   93.098091   165.119348
## K   9574.456216 8526.119846 10751.691682
## phi    7.814186    4.669271    13.077309
\end{verbatim}
\end{kframe}
\end{knitrout}
Plot:
\begin{knitrout}
\definecolor{shadecolor}{rgb}{0.969, 0.969, 0.969}\color{fgcolor}\begin{kframe}
\begin{alltt}
\hlkwd{plot}\hlstd{(logistic_fit,} \hlkwc{level}\hlstd{=}\hlnum{0.95}\hlstd{)}
\end{alltt}
\end{kframe}
\includegraphics[width=\maxwidth]{figure/unnamed-chunk-12-1} 

\end{knitrout}
There's a clear bias in our fit; the estimated trajectory underestimates
the peak epidemic. This is likely to affect our parameter estimates.

We can be smarter about our choices of fitting window. Instead of using
the entire time series, we can fit the logistic model from the
beginning of an epidemic to the next observation after the peak
\citep{ma2014estimating}.
\begin{knitrout}
\definecolor{shadecolor}{rgb}{0.969, 0.969, 0.969}\color{fgcolor}\begin{kframe}
\begin{alltt}
\hlstd{ma_begin} \hlkwb{<-} \hlnum{1}
\hlstd{ma_end} \hlkwb{<-} \hlkwd{which.max}\hlstd{(SierraLeone2014b}\hlopt{$}\hlstd{confirmed)} \hlopt{+} \hlnum{1}

\hlstd{logistic_fit_ma} \hlkwb{<-} \hlkwd{fitode}\hlstd{(}
    \hlstd{logistic_model,}
    \hlkwc{data}\hlstd{=SierraLeone2014b[ma_begin}\hlopt{:}\hlstd{ma_end,],}
    \hlkwc{start}\hlstd{=start_logistic}
\hlstd{)}
\end{alltt}


{\ttfamily\noindent\itshape\color{messagecolor}{\#\# Fitting ode ...}}

{\ttfamily\noindent\itshape\color{messagecolor}{\#\# Computing vcov on the original scale ...}}\end{kframe}
\end{knitrout}

We get a much better fit:
\begin{knitrout}
\definecolor{shadecolor}{rgb}{0.969, 0.969, 0.969}\color{fgcolor}\begin{kframe}
\begin{alltt}
\hlkwd{plot}\hlstd{(logistic_fit,} \hlkwc{level}\hlstd{=}\hlnum{0.95}\hlstd{)}
\hlkwd{plot}\hlstd{(logistic_fit_ma,} \hlkwc{level}\hlstd{=}\hlnum{0.95}\hlstd{,} \hlkwc{add}\hlstd{=}\hlnum{TRUE}\hlstd{,} \hlkwc{col.traj}\hlstd{=}\hlstr{"red"}\hlstd{,} \hlkwc{col.conf}\hlstd{=}\hlstr{"red"}\hlstd{)}
\end{alltt}
\end{kframe}
\includegraphics[width=\maxwidth]{figure/unnamed-chunk-14-1} 

\end{knitrout}

We get slightly wider confidence intervals because we're using less data:
\begin{knitrout}
\definecolor{shadecolor}{rgb}{0.969, 0.969, 0.969}\color{fgcolor}\begin{kframe}
\begin{alltt}
\hlkwd{confint}\hlstd{(logistic_fit_ma)}
\end{alltt}
\begin{verbatim}
##        estimate       2.5 %      97.5 %
## r       9.29878    8.324641    10.38691
## X0    119.49151   86.681357   164.72078
## K   10943.72524 9183.475211 13041.37263
## phi    29.19584   12.515092    68.10952
\end{verbatim}
\end{kframe}
\end{knitrout}

\subsection{SIR model}

The Susceptible-Infected-Recovered (SIR) model describes how disease spreads in
a homogeneous population:
\begin{equation}
\begin{aligned}
\frac{dS}{dt} &= - \beta S \frac{I}{N}\\
\frac{dI}{dt} &= \beta S \frac{I}{N} - \gamma I\\
\frac{dR}{dt} &= \gamma I
\end{aligned}
\end{equation}
We can assume that confirmed cases are put into control and are no longer infectious,
thus effectively recovering from infection \citep{he2009plug};
in other words, we model cumulative number of confirmed cases with
cumulative number of recovered cases (state variable $R$).

Again, we use interval counts by using \texttt{diffnames=``R''}:
\begin{knitrout}
\definecolor{shadecolor}{rgb}{0.969, 0.969, 0.969}\color{fgcolor}\begin{kframe}
\begin{alltt}
\hlstd{SIR_model} \hlkwb{<-} \hlkwd{new}\hlstd{(}\hlstr{"model.ode"}\hlstd{,}
    \hlkwc{name}\hlstd{=}\hlstr{"SIR (nbinom)"}\hlstd{,}
    \hlkwc{model}\hlstd{=}\hlkwd{list}\hlstd{(}
        \hlstd{S} \hlopt{~ -} \hlstd{beta} \hlopt{*} \hlstd{S} \hlopt{*} \hlstd{I}\hlopt{/}\hlstd{N,}
        \hlstd{I} \hlopt{~} \hlstd{beta} \hlopt{*} \hlstd{S} \hlopt{*} \hlstd{I}\hlopt{/}\hlstd{N} \hlopt{-} \hlstd{gamma} \hlopt{*} \hlstd{I,}
        \hlstd{R} \hlopt{~} \hlstd{gamma} \hlopt{*} \hlstd{I}
    \hlstd{),}
    \hlkwc{observation}\hlstd{=}\hlkwd{list}\hlstd{(}
        \hlstd{confirmed} \hlopt{~} \hlkwd{dnbinom}\hlstd{(}\hlkwc{mu}\hlstd{=R,} \hlkwc{size}\hlstd{=phi)}
    \hlstd{),}
    \hlkwc{initial}\hlstd{=}\hlkwd{list}\hlstd{(}
        \hlstd{S} \hlopt{~} \hlstd{N} \hlopt{*} \hlstd{(}\hlnum{1} \hlopt{-} \hlstd{i0),}
        \hlstd{I} \hlopt{~} \hlstd{N} \hlopt{*} \hlstd{i0,}
        \hlstd{R} \hlopt{~} \hlnum{0}
    \hlstd{),}
    \hlkwc{diffnames}\hlstd{=}\hlstr{"R"}\hlstd{,}
    \hlkwc{par}\hlstd{=}\hlkwd{c}\hlstd{(}\hlstr{"beta"}\hlstd{,} \hlstr{"gamma"}\hlstd{,} \hlstr{"N"}\hlstd{,} \hlstr{"i0"}\hlstd{,} \hlstr{"phi"}\hlstd{),}
    \hlkwc{link}\hlstd{=}\hlkwd{c}\hlstd{(}\hlkwc{i0}\hlstd{=}\hlstr{"logit"}\hlstd{)}
\hlstd{)}
\end{alltt}
\end{kframe}
\end{knitrout}
For brevity, we assumed that the initial conditions are given by
\begin{equation}
\begin{aligned}
S(0) &= N (1 - i_0)\\
I(0) &= N i_0\\
R(0) &= 0
\end{aligned}
\end{equation}
where $i_0$ is the initial proportion of infected individuals.
Moreover, setting \texttt{link=c(i0=``logit'')} tells \texttt{fitode} that the
parameter \texttt{i0} needs to be between 0 and 1.

Searching for starting values:
\begin{knitrout}
\definecolor{shadecolor}{rgb}{0.969, 0.969, 0.969}\color{fgcolor}\begin{kframe}
\begin{alltt}
\hlstd{SIR_start} \hlkwb{<-} \hlkwd{c}\hlstd{(}\hlkwc{beta}\hlstd{=}\hlnum{70}\hlstd{,} \hlkwc{gamma}\hlstd{=}\hlnum{60}\hlstd{,} \hlkwc{N}\hlstd{=}\hlnum{40000}\hlstd{,} \hlkwc{i0}\hlstd{=}\hlnum{0.0004}\hlstd{,} \hlkwc{phi}\hlstd{=}\hlnum{6}\hlstd{)}

\hlstd{ss_SIR} \hlkwb{<-} \hlkwd{simulate}\hlstd{(SIR_model,}
    \hlkwc{parms}\hlstd{=SIR_start,} \hlkwc{times}\hlstd{=SierraLeone2014b}\hlopt{$}\hlstd{times)}

\hlkwd{plot}\hlstd{(SierraLeone2014)}
\hlkwd{lines}\hlstd{(ss_SIR}\hlopt{$}\hlstd{times, ss_SIR}\hlopt{$}\hlstd{R)}
\end{alltt}
\end{kframe}
\includegraphics[width=\maxwidth]{figure/unnamed-chunk-17-1} 

\end{knitrout}
Fit:
\begin{knitrout}
\definecolor{shadecolor}{rgb}{0.969, 0.969, 0.969}\color{fgcolor}\begin{kframe}
\begin{alltt}
\hlstd{SIR_fit} \hlkwb{<-} \hlkwd{fitode}\hlstd{(}
    \hlstd{SIR_model,}
    \hlkwc{data}\hlstd{=SierraLeone2014b,}
    \hlkwc{start}\hlstd{=SIR_start}
\hlstd{)}
\end{alltt}


{\ttfamily\noindent\itshape\color{messagecolor}{\#\# Fitting ode ...}}

{\ttfamily\noindent\itshape\color{messagecolor}{\#\# Computing vcov on the original scale ...}}\end{kframe}
\end{knitrout}
Plot:
\begin{knitrout}
\definecolor{shadecolor}{rgb}{0.969, 0.969, 0.969}\color{fgcolor}\begin{kframe}
\begin{alltt}
\hlkwd{plot}\hlstd{(SIR_fit,} \hlkwc{level}\hlstd{=}\hlnum{0.95}\hlstd{)}
\end{alltt}
\end{kframe}
\includegraphics[width=\maxwidth]{figure/sirfitplot-1} 

\end{knitrout}
Again, the SIR model underestimates the peak.

This could be a problem with fitting window.
When we get rid of the long tail in the time series, we get a much better fit:
\begin{knitrout}
\definecolor{shadecolor}{rgb}{0.969, 0.969, 0.969}\color{fgcolor}\begin{kframe}
\begin{alltt}
\hlstd{SIR_fit_b} \hlkwb{<-} \hlkwd{fitode}\hlstd{(}
    \hlstd{SIR_model,}
    \hlkwc{data}\hlstd{=SierraLeone2014b[SierraLeone2014b}\hlopt{$}\hlstd{times} \hlopt{<} \hlnum{2015.4}\hlstd{,],}
    \hlkwc{start}\hlstd{=SIR_start}
\hlstd{)}
\end{alltt}


{\ttfamily\noindent\itshape\color{messagecolor}{\#\# Fitting ode ...}}

{\ttfamily\noindent\itshape\color{messagecolor}{\#\# Computing vcov on the original scale ...}}\begin{alltt}
\hlkwd{plot}\hlstd{(SIR_fit_b,} \hlkwc{level}\hlstd{=}\hlnum{0.95}\hlstd{)}
\end{alltt}
\end{kframe}
\includegraphics[width=\maxwidth]{figure/sirbfit-1} 

\end{knitrout}

There are several ways we can get the confidence intervals on the growth rate
($r = \beta - \gamma$). By default, we can use the delta method (this is the default
option).
\begin{knitrout}
\definecolor{shadecolor}{rgb}{0.969, 0.969, 0.969}\color{fgcolor}\begin{kframe}
\begin{alltt}
\hlkwd{confint}\hlstd{(SIR_fit_b,} \hlkwc{parm}\hlstd{=}\hlkwd{list}\hlstd{(r}\hlopt{~}\hlstd{beta}\hlopt{-}\hlstd{gamma))}
\end{alltt}
\begin{verbatim}
##   estimate    2.5 %  97.5 %
## r  10.4978 9.920495 11.0751
\end{verbatim}
\end{kframe}
\end{knitrout}
We discuss other methods later

\subsection{Summary}

Here, we summarize the estimates from different fits:
\begin{knitrout}
\definecolor{shadecolor}{rgb}{0.969, 0.969, 0.969}\color{fgcolor}\begin{kframe}
\begin{alltt}
\hlstd{fit_summ} \hlkwb{<-} \hlkwd{data.frame}\hlstd{(}
    \hlkwc{fits}\hlstd{=}\hlkwd{c}\hlstd{(}\hlstr{"exponential\textbackslash{}n(poisson)"}\hlstd{,} \hlstr{"exponential\textbackslash{}n(nbinom)"}\hlstd{,}
           \hlstr{"logistic\textbackslash{}n(full)"}\hlstd{,}
           \hlstr{"logistic\textbackslash{}n(window)"}\hlstd{,} \hlstr{"SIR\textbackslash{}n(full)"}\hlstd{,} \hlstr{"SIR\textbackslash{}n(window)"}\hlstd{),}
    \hlkwc{estimate}\hlstd{=}\hlkwd{c}\hlstd{(}\hlkwd{coef}\hlstd{(exp_fit)[}\hlnum{1}\hlstd{],} \hlkwd{coef}\hlstd{(exp_fit_nbinom)[}\hlnum{1}\hlstd{],}
               \hlkwd{coef}\hlstd{(logistic_fit)[}\hlnum{1}\hlstd{],} \hlkwd{coef}\hlstd{(logistic_fit_ma)[}\hlnum{1}\hlstd{],}
               \hlopt{-}\hlkwd{diff}\hlstd{(}\hlkwd{coef}\hlstd{(SIR_fit)[}\hlnum{1}\hlopt{:}\hlnum{2}\hlstd{]),}
               \hlopt{-}\hlkwd{diff}\hlstd{(}\hlkwd{coef}\hlstd{(SIR_fit_b)[}\hlnum{1}\hlopt{:}\hlnum{2}\hlstd{])),}
    \hlkwc{lwr}\hlstd{=}\hlkwd{c}\hlstd{(}\hlkwd{confint}\hlstd{(exp_fit)[}\hlnum{1}\hlstd{,}\hlnum{2}\hlstd{],} \hlkwd{confint}\hlstd{(exp_fit_nbinom)[}\hlnum{1}\hlstd{,}\hlnum{2}\hlstd{],}
          \hlkwd{confint}\hlstd{(logistic_fit)[}\hlnum{1}\hlstd{,}\hlnum{2}\hlstd{],}
          \hlkwd{confint}\hlstd{(logistic_fit_ma)[}\hlnum{1}\hlstd{,}\hlnum{2}\hlstd{],}
          \hlkwd{confint}\hlstd{(SIR_fit,} \hlkwc{parm}\hlstd{=}\hlkwd{list}\hlstd{(r}\hlopt{~}\hlstd{beta}\hlopt{-}\hlstd{gamma))[}\hlnum{2}\hlstd{],}
          \hlkwd{confint}\hlstd{(SIR_fit_b,} \hlkwc{parm}\hlstd{=}\hlkwd{list}\hlstd{(r}\hlopt{~}\hlstd{beta}\hlopt{-}\hlstd{gamma))[}\hlnum{2}\hlstd{]),}
    \hlkwc{upr}\hlstd{=}\hlkwd{c}\hlstd{(}\hlkwd{confint}\hlstd{(exp_fit)[}\hlnum{1}\hlstd{,}\hlnum{3}\hlstd{],} \hlkwd{confint}\hlstd{(exp_fit_nbinom)[}\hlnum{1}\hlstd{,}\hlnum{3}\hlstd{],}
          \hlkwd{confint}\hlstd{(logistic_fit)[}\hlnum{1}\hlstd{,}\hlnum{3}\hlstd{],}
          \hlkwd{confint}\hlstd{(logistic_fit_ma)[}\hlnum{1}\hlstd{,}\hlnum{3}\hlstd{],}
          \hlkwd{confint}\hlstd{(SIR_fit,} \hlkwc{parm}\hlstd{=}\hlkwd{list}\hlstd{(r}\hlopt{~}\hlstd{beta}\hlopt{-}\hlstd{gamma))[}\hlnum{3}\hlstd{],}
          \hlkwd{confint}\hlstd{(SIR_fit_b,} \hlkwc{parm}\hlstd{=}\hlkwd{list}\hlstd{(r}\hlopt{~}\hlstd{beta}\hlopt{-}\hlstd{gamma))[}\hlnum{3}\hlstd{])}
\hlstd{)}

\hlstd{fit_summ}\hlopt{$}\hlstd{fits} \hlkwb{<-} \hlkwd{factor}\hlstd{(fit_summ}\hlopt{$}\hlstd{fits,} \hlkwc{level}\hlstd{=fit_summ}\hlopt{$}\hlstd{fits)}

\hlkwd{ggplot}\hlstd{(fit_summ)} \hlopt{+}
    \hlkwd{geom_point}\hlstd{(}\hlkwd{aes}\hlstd{(fits, estimate))} \hlopt{+}
    \hlkwd{geom_errorbar}\hlstd{(}\hlkwd{aes}\hlstd{(fits,} \hlkwc{ymin}\hlstd{=lwr,} \hlkwc{ymax}\hlstd{=upr),} \hlkwc{width}\hlstd{=}\hlnum{0.2}\hlstd{)} \hlopt{+}
    \hlkwd{scale_x_discrete}\hlstd{(}\hlstr{"Models"}\hlstd{)} \hlopt{+}
    \hlkwd{scale_y_continuous}\hlstd{(}\hlstr{"Initial epidemic growth rate"}\hlstd{)}
\end{alltt}
\end{kframe}
\includegraphics[width=\maxwidth]{figure/fitsummary-1} 

\end{knitrout}

\bibliography{fitode}
\end{document}
