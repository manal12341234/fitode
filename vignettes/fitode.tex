%\VignetteEngine{knitr::knitr}
%\VignetteDepends{ggplot2}
%\VignetteDepends{plyr}
%\VignetteDepends{dplyr}
%\VignetteDepends{reshape2}
%\VignetteIndexEntry{Basic ODE model fitting}
\documentclass{article}\usepackage[]{graphicx}\usepackage[]{color}
%% maxwidth is the original width if it is less than linewidth
%% otherwise use linewidth (to make sure the graphics do not exceed the margin)
\makeatletter
\def\maxwidth{ %
  \ifdim\Gin@nat@width>\linewidth
    \linewidth
  \else
    \Gin@nat@width
  \fi
}
\makeatother

\definecolor{fgcolor}{rgb}{0.345, 0.345, 0.345}
\newcommand{\hlnum}[1]{\textcolor[rgb]{0.686,0.059,0.569}{#1}}%
\newcommand{\hlstr}[1]{\textcolor[rgb]{0.192,0.494,0.8}{#1}}%
\newcommand{\hlcom}[1]{\textcolor[rgb]{0.678,0.584,0.686}{\textit{#1}}}%
\newcommand{\hlopt}[1]{\textcolor[rgb]{0,0,0}{#1}}%
\newcommand{\hlstd}[1]{\textcolor[rgb]{0.345,0.345,0.345}{#1}}%
\newcommand{\hlkwa}[1]{\textcolor[rgb]{0.161,0.373,0.58}{\textbf{#1}}}%
\newcommand{\hlkwb}[1]{\textcolor[rgb]{0.69,0.353,0.396}{#1}}%
\newcommand{\hlkwc}[1]{\textcolor[rgb]{0.333,0.667,0.333}{#1}}%
\newcommand{\hlkwd}[1]{\textcolor[rgb]{0.737,0.353,0.396}{\textbf{#1}}}%
\let\hlipl\hlkwb

\usepackage{framed}
\makeatletter
\newenvironment{kframe}{%
 \def\at@end@of@kframe{}%
 \ifinner\ifhmode%
  \def\at@end@of@kframe{\end{minipage}}%
  \begin{minipage}{\columnwidth}%
 \fi\fi%
 \def\FrameCommand##1{\hskip\@totalleftmargin \hskip-\fboxsep
 \colorbox{shadecolor}{##1}\hskip-\fboxsep
     % There is no \\@totalrightmargin, so:
     \hskip-\linewidth \hskip-\@totalleftmargin \hskip\columnwidth}%
 \MakeFramed {\advance\hsize-\width
   \@totalleftmargin\z@ \linewidth\hsize
   \@setminipage}}%
 {\par\unskip\endMakeFramed%
 \at@end@of@kframe}
\makeatother

\definecolor{shadecolor}{rgb}{.97, .97, .97}
\definecolor{messagecolor}{rgb}{0, 0, 0}
\definecolor{warningcolor}{rgb}{1, 0, 1}
\definecolor{errorcolor}{rgb}{1, 0, 0}
\newenvironment{knitrout}{}{} % an empty environment to be redefined in TeX

\usepackage{alltt}
\title{Basic ODE fitting}
\usepackage{amsmath}
\usepackage{natbib}
\usepackage{hyperref}
\newcommand{\rzero}{{\cal R}_0}
\newcommand{\code}[1]{{\tt #1}}
\bibliographystyle{chicago}
\date{\today}
\IfFileExists{upquote.sty}{\usepackage{upquote}}{}
\begin{document}
\maketitle



\tableofcontents

\section{Preliminaries}

Load packages:

\begin{knitrout}
\definecolor{shadecolor}{rgb}{0.969, 0.969, 0.969}\color{fgcolor}\begin{kframe}
\begin{alltt}
\hlkwd{library}\hlstd{(fitode)}
\end{alltt}
\end{kframe}
\end{knitrout}

\section{Basic fitting}

\subsection{Exponential decay model}

Suppose we have a quantity that is decreasing exponentially.

\begin{knitrout}
\definecolor{shadecolor}{rgb}{0.969, 0.969, 0.969}\color{fgcolor}\begin{kframe}
\begin{alltt}
\hlkwd{set.seed}\hlstd{(}\hlnum{123}\hlstd{)}
\hlstd{true.m} \hlkwb{<-} \hlnum{0.5}
\hlstd{true.A0} \hlkwb{<-} \hlnum{200}
\hlstd{true.sd} \hlkwb{<-} \hlnum{15}

\hlstd{exp.data} \hlkwb{<-} \hlkwd{data.frame}\hlstd{(}
    \hlkwc{x}\hlstd{=}\hlkwd{rep}\hlstd{(}\hlnum{0}\hlopt{:}\hlnum{4}\hlstd{,} \hlnum{3}\hlstd{),}
    \hlkwc{y}\hlstd{=}\hlkwd{rnorm}\hlstd{(}\hlnum{15}\hlstd{, true.A0} \hlopt{*} \hlkwd{exp}\hlstd{(}\hlopt{-}\hlstd{true.m} \hlopt{*} \hlkwd{rep}\hlstd{(}\hlnum{0}\hlopt{:}\hlnum{4}\hlstd{,} \hlnum{3}\hlstd{)),} \hlkwc{sd}\hlstd{=true.sd)}
\hlstd{)}

\hlkwd{plot}\hlstd{(exp.data)}
\end{alltt}
\end{kframe}
\includegraphics[width=\maxwidth]{figure/unnamed-chunk-1-1} 

\end{knitrout}

The true dynamics can be modeled with the following equation:
$$
\frac{dA}{dt} = - m A
$$

We can translate this into a \code{fitode} model as follows:

\begin{knitrout}
\definecolor{shadecolor}{rgb}{0.969, 0.969, 0.969}\color{fgcolor}\begin{kframe}
\begin{alltt}
\hlstd{exp.model} \hlkwb{<-} \hlkwd{new}\hlstd{(}\hlstr{"model.ode"}\hlstd{,}
    \hlkwc{name} \hlstd{=} \hlstr{"SI"}\hlstd{,}
    \hlkwc{model} \hlstd{=} \hlkwd{list}\hlstd{(}
        \hlstd{A} \hlopt{~ -}\hlstd{m} \hlopt{*} \hlstd{A}
    \hlstd{),}
    \hlkwc{observation} \hlstd{=} \hlkwd{list}\hlstd{(}
        \hlstd{y} \hlopt{~} \hlkwd{dnorm}\hlstd{(}\hlkwc{mean}\hlstd{=A,} \hlkwc{sd}\hlstd{=sd)}
    \hlstd{),}
    \hlkwc{initial} \hlstd{=} \hlkwd{list}\hlstd{(}
        \hlstd{A} \hlopt{~} \hlstd{A0}
    \hlstd{),}
    \hlkwc{par}\hlstd{=}\hlkwd{c}\hlstd{(}\hlstr{"m"}\hlstd{,} \hlstr{"A0"}\hlstd{,} \hlstr{"sd"}\hlstd{)}
\hlstd{)}
\end{alltt}
\end{kframe}
\end{knitrout}

Then, we can fit the model:

\begin{knitrout}
\definecolor{shadecolor}{rgb}{0.969, 0.969, 0.969}\color{fgcolor}\begin{kframe}
\begin{alltt}
\hlstd{exp.fit} \hlkwb{<-} \hlkwd{fitode}\hlstd{(}
    \hlstd{exp.model,}
    \hlstd{exp.data,}
    \hlkwc{start}\hlstd{=}\hlkwd{c}\hlstd{(}\hlkwc{m}\hlstd{=}\hlnum{0.5}\hlstd{,} \hlkwc{A0}\hlstd{=}\hlnum{200}\hlstd{,} \hlkwc{sd}\hlstd{=}\hlnum{15}\hlstd{),}
    \hlkwc{tcol}\hlstd{=}\hlstr{"x"}
\hlstd{)}
\end{alltt}


{\ttfamily\noindent\itshape\color{messagecolor}{\#\# Fitting ode ...}}

{\ttfamily\noindent\itshape\color{messagecolor}{\#\# Computing vcov on the original scale ...}}\end{kframe}
\end{knitrout}

To diagnose the fit, we can use \code{plot} function.
Using the \code{level} argument will plot 95\% confidence intervals of the true trajectory, estimated via delta method.

\begin{knitrout}
\definecolor{shadecolor}{rgb}{0.969, 0.969, 0.969}\color{fgcolor}\begin{kframe}
\begin{alltt}
\hlkwd{plot}\hlstd{(exp.fit,} \hlkwc{level}\hlstd{=}\hlnum{0.95}\hlstd{)}
\hlkwd{curve}\hlstd{(true.A0} \hlopt{*} \hlkwd{exp}\hlstd{(}\hlopt{-}\hlstd{true.m}\hlopt{*}\hlstd{x),} \hlkwc{add}\hlstd{=}\hlnum{TRUE}\hlstd{,} \hlkwc{lty}\hlstd{=}\hlnum{1}\hlstd{,} \hlkwc{col}\hlstd{=}\hlstr{"red"}\hlstd{)}
\hlkwd{legend}\hlstd{(}
    \hlkwc{x}\hlstd{=}\hlstr{"topright"}\hlstd{,}
    \hlkwc{legend}\hlstd{=}\hlkwd{c}\hlstd{(}\hlstr{"estimated"}\hlstd{,} \hlstr{"true"}\hlstd{),}
    \hlkwc{col}\hlstd{=}\hlkwd{c}\hlstd{(}\hlstr{"black"}\hlstd{,} \hlstr{"red"}\hlstd{),}
    \hlkwc{lty}\hlstd{=}\hlnum{1}
\hlstd{)}
\end{alltt}
\end{kframe}
\includegraphics[width=\maxwidth]{figure/unnamed-chunk-4-1} 

\end{knitrout}

To obtain the confidence interval, we can use \code{confint} function.
There are three available methods for obtaining the confidence intervals: \code{delta}, \code{profile} and \code{wmvrnorm}.
Due to computation speed, default option is \code{delta}.
We will get into the details later.
For now,

\begin{knitrout}
\definecolor{shadecolor}{rgb}{0.969, 0.969, 0.969}\color{fgcolor}\begin{kframe}
\begin{alltt}
\hlkwd{confint}\hlstd{(exp.fit)}
\end{alltt}
\begin{verbatim}
##       estimate       2.5 %      97.5 %
## m    0.5303757   0.4757616   0.5912593
## A0 212.1596895 200.6325622 224.3490954
## sd  11.0526632   7.7278640  15.8079080
\end{verbatim}
\end{kframe}
\end{knitrout}

Note that in this particular example, we can use \code{glm} function to fit the model as well.
We can compare the results.

\begin{knitrout}
\definecolor{shadecolor}{rgb}{0.969, 0.969, 0.969}\color{fgcolor}\begin{kframe}
\begin{alltt}
\hlstd{glm.fit} \hlkwb{<-} \hlkwd{glm}\hlstd{(y}\hlopt{~}\hlstd{x,}
    \hlkwc{family}\hlstd{=}\hlkwd{gaussian}\hlstd{(}\hlkwc{link}\hlstd{=}\hlstr{"log"}\hlstd{),}
    \hlkwc{data}\hlstd{=exp.data,}
    \hlkwc{start} \hlstd{=} \hlkwd{c}\hlstd{(}\hlkwc{intercept}\hlstd{=}\hlkwd{log}\hlstd{(}\hlnum{200}\hlstd{),} \hlkwc{x}\hlstd{=}\hlopt{-}\hlnum{0.5}\hlstd{))}

\hlstd{glm.pred} \hlkwb{<-} \hlkwd{predict}\hlstd{(glm.fit,} \hlkwd{data.frame}\hlstd{(}\hlkwc{x}\hlstd{=}\hlnum{0}\hlopt{:}\hlnum{4}\hlstd{),} \hlkwc{se.fit}\hlstd{=}\hlnum{TRUE}\hlstd{,} \hlkwc{type}\hlstd{=}\hlstr{"response"}\hlstd{)}

\hlstd{glm.data} \hlkwb{<-} \hlkwd{data.frame}\hlstd{(}
    \hlkwc{x}\hlstd{=}\hlnum{0}\hlopt{:}\hlnum{4}\hlstd{,}
    \hlkwc{estimate}\hlstd{=glm.pred}\hlopt{$}\hlstd{fit,}
    \hlkwc{lwr}\hlstd{=glm.pred}\hlopt{$}\hlstd{fit}\hlopt{-}\hlnum{1.96} \hlopt{*} \hlstd{glm.pred}\hlopt{$}\hlstd{se.fit,}
    \hlkwc{upr}\hlstd{=glm.pred}\hlopt{$}\hlstd{fit}\hlopt{+}\hlnum{1.96} \hlopt{*} \hlstd{glm.pred}\hlopt{$}\hlstd{se.fit}
\hlstd{)}

\hlkwd{plot}\hlstd{(exp.fit,} \hlkwc{level}\hlstd{=}\hlnum{0.95}\hlstd{)}
\hlkwd{lines}\hlstd{(glm.data}\hlopt{$}\hlstd{x, glm.data}\hlopt{$}\hlstd{estimate,} \hlkwc{col}\hlstd{=}\hlnum{2}\hlstd{)}
\hlkwd{lines}\hlstd{(glm.data}\hlopt{$}\hlstd{x, glm.data}\hlopt{$}\hlstd{lwr,} \hlkwc{col}\hlstd{=}\hlnum{2}\hlstd{)}
\hlkwd{lines}\hlstd{(glm.data}\hlopt{$}\hlstd{x, glm.data}\hlopt{$}\hlstd{upr,} \hlkwc{col}\hlstd{=}\hlnum{2}\hlstd{)}
\hlkwd{legend}\hlstd{(}
    \hlkwc{x}\hlstd{=}\hlstr{"topright"}\hlstd{,}
    \hlkwc{legend}\hlstd{=}\hlkwd{c}\hlstd{(}\hlstr{"estimated (fitode)"}\hlstd{,} \hlstr{"estimated (glm)"}\hlstd{),}
    \hlkwc{col}\hlstd{=}\hlkwd{c}\hlstd{(}\hlstr{"black"}\hlstd{,} \hlstr{"red"}\hlstd{),}
    \hlkwc{lty}\hlstd{=}\hlnum{1}
\hlstd{)}
\end{alltt}
\end{kframe}
\includegraphics[width=\maxwidth]{figure/unnamed-chunk-6-1} 

\end{knitrout}

Estimated trajectories and their confidence intervals are essentially indistinguishable.

\subsection{Chemical reaction - multiple state fitting}

Now, consider the following chemical reaction:
$$
A \to 3 B
$$
Then, we can write the governing differential equation as follows:
$$
\begin{aligned}
\frac{dA}{dt} &= - k A\\
\frac{dB}{dt} &= 3 k B
\end{aligned}
$$
Suppose we have measured both quantities and have data:


\begin{knitrout}
\definecolor{shadecolor}{rgb}{0.969, 0.969, 0.969}\color{fgcolor}\begin{kframe}
\begin{alltt}
\hlkwd{head}\hlstd{(reaction_data)}
\end{alltt}
\begin{verbatim}
##   times       y1       y2
## 1     1 317.8691 106.8864
## 2     2 276.4298 191.1854
## 3     3 225.9531 262.5231
## 4     4 229.2591 330.2038
## 5     5 196.3682 392.9070
## 6     6 171.2810 447.1751
\end{verbatim}
\end{kframe}
\end{knitrout}

Here, \code{y1} meausres quantity $A$ and \code{y2} measures quantity $B$.
Then, we can define the model:

\begin{knitrout}
\definecolor{shadecolor}{rgb}{0.969, 0.969, 0.969}\color{fgcolor}\begin{kframe}
\begin{alltt}
\hlstd{reaction_model} \hlkwb{<-} \hlkwd{new}\hlstd{(}\hlstr{"model.ode"}\hlstd{,}
    \hlkwc{name} \hlstd{=} \hlstr{"reaction"}\hlstd{,}
    \hlkwc{model} \hlstd{=} \hlkwd{list}\hlstd{(}
        \hlstd{A} \hlopt{~ -} \hlstd{k} \hlopt{*} \hlstd{A,}
        \hlstd{B} \hlopt{~} \hlnum{3} \hlopt{*} \hlstd{k} \hlopt{*} \hlstd{A}
    \hlstd{),}
    \hlkwc{observation} \hlstd{=} \hlkwd{list}\hlstd{(}
        \hlstd{y1} \hlopt{~} \hlkwd{dnorm}\hlstd{(}\hlkwc{mean}\hlstd{=A,} \hlkwc{sd}\hlstd{=sd),}
        \hlstd{y2} \hlopt{~} \hlkwd{dnorm}\hlstd{(}\hlkwc{mean}\hlstd{=B,} \hlkwc{sd}\hlstd{=sd)}
    \hlstd{),}
    \hlkwc{initial} \hlstd{=} \hlkwd{list}\hlstd{(}
        \hlstd{A}\hlopt{~}\hlstd{A0,}
        \hlstd{B}\hlopt{~}\hlstd{B0}
    \hlstd{),}
    \hlkwc{par}\hlstd{=}\hlkwd{c}\hlstd{(}\hlstr{"k"}\hlstd{,} \hlstr{"A0"}\hlstd{,} \hlstr{"B0"}\hlstd{,} \hlstr{"sd"}\hlstd{)}
\hlstd{)}
\end{alltt}
\end{kframe}
\end{knitrout}

We can fit this using arbitrary starting conditions.

\begin{knitrout}
\definecolor{shadecolor}{rgb}{0.969, 0.969, 0.969}\color{fgcolor}\begin{kframe}
\begin{alltt}
\hlstd{reaction_fit} \hlkwb{<-} \hlkwd{fitode}\hlstd{(}
    \hlstd{reaction_model,}
    \hlstd{reaction_data,}
    \hlkwc{start}\hlstd{=}\hlkwd{c}\hlstd{(}\hlkwc{k}\hlstd{=}\hlnum{0.1}\hlstd{,} \hlkwc{A0}\hlstd{=}\hlnum{300}\hlstd{,} \hlkwc{B0}\hlstd{=}\hlnum{10}\hlstd{,} \hlkwc{sd}\hlstd{=}\hlnum{10}\hlstd{)}
\hlstd{)}
\end{alltt}


{\ttfamily\noindent\itshape\color{messagecolor}{\#\# Fitting ode ...}}

{\ttfamily\noindent\itshape\color{messagecolor}{\#\# Computing vcov on the original scale ...}}\begin{alltt}
\hlkwd{plot}\hlstd{(reaction_fit,} \hlkwc{level}\hlstd{=}\hlnum{0.95}\hlstd{)}
\end{alltt}
\end{kframe}
\includegraphics[width=\maxwidth]{figure/unnamed-chunk-10-1} 

\end{knitrout}

Confidence intervals...

\begin{knitrout}
\definecolor{shadecolor}{rgb}{0.969, 0.969, 0.969}\color{fgcolor}\begin{kframe}
\begin{alltt}
\hlkwd{confint}\hlstd{(reaction_fit)}
\end{alltt}
\begin{verbatim}
##      estimate        2.5 %      97.5 %
## k    0.100314   0.09714102   0.1035907
## A0 299.581692 295.20809184 304.0200894
## B0 100.734330  90.39960933 112.2505441
## sd   9.151578   7.35070035  11.3936588
\end{verbatim}
\end{kframe}
\end{knitrout}

\subsection{Fitting SIR model}

\begin{knitrout}
\definecolor{shadecolor}{rgb}{0.969, 0.969, 0.969}\color{fgcolor}\begin{kframe}
\begin{alltt}
\hlstd{harbin} \hlkwb{<-} \hlstd{fitsir}\hlopt{::}\hlstd{harbin}
\hlstd{harbin2} \hlkwb{<-} \hlkwd{rbind}\hlstd{(}\hlkwd{data.frame}\hlstd{(}\hlkwc{week}\hlstd{=}\hlnum{1}\hlstd{,} \hlkwc{Deaths}\hlstd{=}\hlnum{NA}\hlstd{), harbin)}

\hlkwd{plot}\hlstd{(harbin2)}
\end{alltt}
\end{kframe}
\includegraphics[width=\maxwidth]{figure/unnamed-chunk-12-1} 

\end{knitrout}

We need to add NA observation to make this work...

\begin{knitrout}
\definecolor{shadecolor}{rgb}{0.969, 0.969, 0.969}\color{fgcolor}\begin{kframe}
\begin{alltt}
\hlstd{SI_model_c} \hlkwb{<-} \hlkwd{new}\hlstd{(}\hlstr{"model.ode"}\hlstd{,}
    \hlkwc{name} \hlstd{=} \hlstr{"SI"}\hlstd{,}
    \hlkwc{model} \hlstd{=} \hlkwd{list}\hlstd{(}
        \hlstd{S} \hlopt{~ -} \hlstd{beta}\hlopt{*}\hlstd{S}\hlopt{*}\hlstd{I}\hlopt{/}\hlstd{N,}
        \hlstd{I} \hlopt{~} \hlstd{beta}\hlopt{*}\hlstd{S}\hlopt{*}\hlstd{I}\hlopt{/}\hlstd{N} \hlopt{-} \hlstd{gamma}\hlopt{*}\hlstd{I,}
        \hlstd{cDeath} \hlopt{~} \hlstd{gamma}\hlopt{*}\hlstd{I}
    \hlstd{),}
    \hlkwc{observation} \hlstd{=} \hlkwd{list}\hlstd{(}
        \hlstd{Deaths} \hlopt{~} \hlkwd{dnbinom}\hlstd{(}\hlkwc{mu}\hlstd{=cDeath,} \hlkwc{size}\hlstd{=size)}
    \hlstd{),}
    \hlkwc{initial} \hlstd{=} \hlkwd{list}\hlstd{(}
        \hlstd{S} \hlopt{~} \hlstd{N} \hlopt{*} \hlstd{(}\hlnum{1} \hlopt{-} \hlstd{i0),}
        \hlstd{I} \hlopt{~} \hlstd{N} \hlopt{*} \hlstd{i0,}
        \hlstd{cDeath} \hlopt{~} \hlnum{0}
    \hlstd{),}
    \hlkwc{diffnames}\hlstd{=}\hlstr{"cDeath"}\hlstd{,}
    \hlkwc{par}\hlstd{=}\hlkwd{c}\hlstd{(}\hlstr{"beta"}\hlstd{,} \hlstr{"gamma"}\hlstd{,} \hlstr{"N"}\hlstd{,} \hlstr{"i0"}\hlstd{,} \hlstr{"size"}\hlstd{)}
\hlstd{)}

\hlstd{start} \hlkwb{<-} \hlkwd{c}\hlstd{(}\hlkwc{beta}\hlstd{=}\hlnum{2}\hlstd{,} \hlkwc{gamma}\hlstd{=}\hlnum{1}\hlstd{,} \hlkwc{N}\hlstd{=}\hlnum{20000}\hlstd{,} \hlkwc{i0}\hlstd{=}\hlnum{1e-5}\hlstd{,} \hlkwc{size}\hlstd{=}\hlnum{10}\hlstd{)}

\hlstd{sirfit} \hlkwb{<-} \hlkwd{fitode}\hlstd{(}
    \hlstd{SI_model_c,}
    \hlstd{harbin2,}
    \hlkwc{start}\hlstd{=start,}
    \hlkwc{link} \hlstd{=} \hlkwd{list}\hlstd{(}
        \hlkwc{beta}\hlstd{=}\hlstr{"log"}\hlstd{,}
        \hlkwc{gamma}\hlstd{=}\hlstr{"log"}\hlstd{,}
        \hlkwc{N}\hlstd{=}\hlstr{"log"}\hlstd{,}
        \hlkwc{i0}\hlstd{=}\hlstr{"logit"}\hlstd{,}
        \hlkwc{size}\hlstd{=}\hlstr{"log"}
    \hlstd{),}
    \hlkwc{tcol}\hlstd{=}\hlstr{"week"}
\hlstd{)}
\end{alltt}


{\ttfamily\noindent\itshape\color{messagecolor}{\#\# Fitting ode ...}}

{\ttfamily\noindent\itshape\color{messagecolor}{\#\# Computing vcov on the original scale ...}}\begin{alltt}
\hlkwd{plot}\hlstd{(sirfit,} \hlkwc{level}\hlstd{=}\hlnum{0.95}\hlstd{)}
\end{alltt}
\end{kframe}
\includegraphics[width=\maxwidth]{figure/sirfit-1} 

\end{knitrout}

Confidence intervals on various epidemiological quantities:

\begin{knitrout}
\definecolor{shadecolor}{rgb}{0.969, 0.969, 0.969}\color{fgcolor}\begin{kframe}
\begin{alltt}
\hlkwd{confint}\hlstd{(sirfit,}
        \hlkwc{parm}\hlstd{=}\hlkwd{list}\hlstd{(}
            \hlstd{R0}\hlopt{~}\hlstd{beta}\hlopt{/}\hlstd{gamma,}
            \hlstd{r}\hlopt{~}\hlstd{beta}\hlopt{-}\hlstd{gamma}
        \hlstd{),}
        \hlkwc{method}\hlstd{=}\hlstr{"wmvrnorm"}\hlstd{)}
\end{alltt}
\begin{verbatim}
##    estimate     2.5 %   97.5 %
## R0 1.705258 1.4733837 2.132810
## r  0.772026 0.6942409 0.837602
\end{verbatim}
\end{kframe}
\end{knitrout}

Alternate parameterization:

\begin{knitrout}
\definecolor{shadecolor}{rgb}{0.969, 0.969, 0.969}\color{fgcolor}\begin{kframe}
\begin{alltt}
\hlstd{SI_model_c2} \hlkwb{<-} \hlkwd{Transform}\hlstd{(}
    \hlstd{SI_model_c,}
    \hlkwd{list}\hlstd{(}
        \hlstd{beta}\hlopt{~}\hlstd{(R0_1}\hlopt{+}\hlnum{1}\hlstd{)}\hlopt{*}\hlstd{gamma}
    \hlstd{),}
    \hlkwc{par}\hlstd{=}\hlkwd{c}\hlstd{(}\hlstr{"R0_1"}\hlstd{,} \hlstr{"gamma"}\hlstd{,} \hlstr{"N"}\hlstd{,} \hlstr{"i0"}\hlstd{,} \hlstr{"size"}\hlstd{)}
\hlstd{)}

\hlstd{cc} \hlkwb{<-} \hlkwd{coef}\hlstd{(sirfit)}

\hlstd{start2} \hlkwb{<-} \hlkwd{c}\hlstd{(}\hlkwc{R0_1}\hlstd{=}\hlkwd{unname}\hlstd{(cc[}\hlnum{1}\hlstd{]}\hlopt{/}\hlstd{cc[}\hlnum{2}\hlstd{]}\hlopt{-}\hlnum{1}\hlstd{), cc[}\hlopt{-}\hlnum{1}\hlstd{])}

\hlstd{sirfit2} \hlkwb{<-} \hlkwd{fitode}\hlstd{(}
    \hlstd{SI_model_c2,}
    \hlstd{harbin2,}
    \hlkwc{start}\hlstd{=start2,}
    \hlkwc{link} \hlstd{=} \hlkwd{list}\hlstd{(}
        \hlkwc{R0_1}\hlstd{=}\hlstr{"log"}\hlstd{,}
        \hlkwc{gamma}\hlstd{=}\hlstr{"log"}\hlstd{,}
        \hlkwc{N}\hlstd{=}\hlstr{"log"}\hlstd{,}
        \hlkwc{i0}\hlstd{=}\hlstr{"logit"}\hlstd{,}
        \hlkwc{size}\hlstd{=}\hlstr{"log"}
    \hlstd{),}
    \hlkwc{tcol}\hlstd{=}\hlstr{"week"}
\hlstd{)}
\end{alltt}


{\ttfamily\noindent\itshape\color{messagecolor}{\#\# Fitting ode ...}}

{\ttfamily\noindent\itshape\color{messagecolor}{\#\# Computing vcov on the original scale ...}}\end{kframe}
\end{knitrout}

Compare fits:

\begin{knitrout}
\definecolor{shadecolor}{rgb}{0.969, 0.969, 0.969}\color{fgcolor}\begin{kframe}
\begin{alltt}
\hlkwd{plot}\hlstd{(sirfit,} \hlkwc{level}\hlstd{=}\hlnum{0.95}\hlstd{)}
\hlkwd{plot}\hlstd{(sirfit2,} \hlkwc{level}\hlstd{=}\hlnum{0.95}\hlstd{,} \hlkwc{add}\hlstd{=}\hlnum{TRUE}\hlstd{,} \hlkwc{col.traj}\hlstd{=}\hlstr{"red"}\hlstd{,} \hlkwc{col.conf}\hlstd{=}\hlstr{"red"}\hlstd{)}
\end{alltt}
\end{kframe}
\includegraphics[width=\maxwidth]{figure/unnamed-chunk-14-1} 

\end{knitrout}

We get identical fits. The advantage of this parameterization is that we can obtain profile confidence intervals on R0 (it's a little slow...):

\begin{knitrout}
\definecolor{shadecolor}{rgb}{0.969, 0.969, 0.969}\color{fgcolor}\begin{kframe}
\begin{alltt}
\hlkwd{set.seed}\hlstd{(}\hlnum{101}\hlstd{)}
\hlkwd{confint}\hlstd{(sirfit2,} \hlstr{"R0_1"}\hlstd{,} \hlkwc{method}\hlstd{=}\hlstr{"profile"}\hlstd{)} \hlopt{+} \hlnum{1}
\end{alltt}
\begin{verbatim}
##      estimate    2.5 %   97.5 %
## R0_1 1.705259 1.000006 2.591918
\end{verbatim}
\begin{alltt}
\hlkwd{confint}\hlstd{(sirfit2,} \hlstr{"R0_1"}\hlstd{,} \hlkwc{method}\hlstd{=}\hlstr{"wmvrnorm"}\hlstd{)} \hlopt{+} \hlnum{1}
\end{alltt}
\begin{verbatim}
##      estimate    2.5 %   97.5 %
## R0_1 1.705259 1.238593 2.437111
\end{verbatim}
\begin{alltt}
\hlkwd{confint}\hlstd{(sirfit,} \hlkwc{parm}\hlstd{=}\hlkwd{list}\hlstd{(R0}\hlopt{~}\hlstd{beta}\hlopt{/}\hlstd{gamma))}
\end{alltt}
\begin{verbatim}
##    estimate     2.5 %   97.5 %
## R0 1.705258 0.9361381 2.474379
\end{verbatim}
\begin{alltt}
\hlkwd{confint}\hlstd{(sirfit,} \hlkwc{parm}\hlstd{=}\hlkwd{list}\hlstd{(R0}\hlopt{~}\hlstd{beta}\hlopt{/}\hlstd{gamma),} \hlkwc{method}\hlstd{=}\hlstr{"wmvrnorm"}\hlstd{)}
\end{alltt}
\begin{verbatim}
##    estimate   2.5 %   97.5 %
## R0 1.705258 1.47908 2.147964
\end{verbatim}
\end{kframe}
\end{knitrout}

I'm not sure why performing wmvrnorm on beta gamma scale gives narrower confidence intervals. Maybe it's because of the parameterization?

\begin{knitrout}
\definecolor{shadecolor}{rgb}{0.969, 0.969, 0.969}\color{fgcolor}\begin{kframe}
\begin{alltt}
\hlkwd{set.seed}\hlstd{(}\hlnum{101}\hlstd{)}
\hlkwd{plot}\hlstd{(sirfit,} \hlkwc{level}\hlstd{=}\hlnum{0.95}\hlstd{,} \hlkwc{method}\hlstd{=}\hlstr{"wmvrnorm"}\hlstd{)}
\hlkwd{plot}\hlstd{(sirfit2,} \hlkwc{level}\hlstd{=}\hlnum{0.95}\hlstd{,} \hlkwc{add}\hlstd{=}\hlnum{TRUE}\hlstd{,} \hlkwc{col.traj}\hlstd{=}\hlstr{"red"}\hlstd{,} \hlkwc{col.conf}\hlstd{=}\hlstr{"red"}\hlstd{,} \hlkwc{method}\hlstd{=}\hlstr{"wmvrnorm"}\hlstd{)}
\end{alltt}
\end{kframe}
\includegraphics[width=\maxwidth]{figure/unnamed-chunk-15-1} 

\end{knitrout}

\end{document}
